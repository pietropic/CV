% CV Pietro Mesquita Piccione
% Dernière mise à jour : 02.04.2020

\documentclass[11pt,a4paper,sans]{moderncv}
\fancyfoot[l]{\parbox[b]{10cm}{\color{color2}\addressfont\strut Pietro{}~Piccione{}~~\emailsymbol\emaillink{pietropic@pm.me}}}
\usepackage[utf8]{inputenc}
\usepackage[polutonikogreek,french]{babel}
\usepackage{dsfont}

\moderncvcolor{orange}
\moderncvstyle{classic}

\usepackage[scale=0.9]{geometry}


%----------------------------------------------------------------------------------------
%	INFORMATIONS PERSONNELLES
%----------------------------------------------------------------------------------------

\firstname{Pietro}
\familyname{Piccione}

\title{Résumé}
\email{pietropic@pm.me}
\homepage{pietropic.github.io/}

%----------------------------------------------------------------------------------------



\begin{document}

\makecvtitle

\section{\enspace Expérience professionnelle}
\cventry{2025--2027}{Postdoctorat}{Université de Göteborg}{Göteborg, Suède}{Encadrant : David Witt Nyström}{}

\section{\enspace Formation}
\cventry{2022 -- 2025}{\emph{Doctorat en Mathématiques}}{Institut des mathématiques de Jussieu}{Sorbonne Université}{Paris, France}{Directeurs : Sébastien Boucksom et Tat Dat Tô}{}
\cventry{2021 -- 2022}{\emph{Master en Mathématiques}}{Institut des mathématiques de Jussieu}{Sorbonne Université}{Paris, France}{}{}
\cventry{2017 -- 2021}{\emph{Licence en Mathématiques}}{Instituto de Matemática e Estatística}{Universidade de São Paulo}{São Paulo, Brésil}{Prix d'excellence académique}{}

\section{\enspace Projets de recherche et financements}
\cventry{2022 -- 2025}{MathInParis2020}{Bourse de doctorat à Paris}{Cofinancé par les actions Marie Sklodowska‑Curie}{Fondation sciences mathématiques de Paris}{}
\cventry {2021 -- 2022}{Bourse PGSM}{Bourse de master à Paris}{Fondation sciences mathématiques de Paris}{}{}
\cventry{2020 -- 2021}{Théorie de Morse}{Projet de recherche de premier cycle}{FAPESP}{}{\href{https://bv.fapesp.br/en/bolsas/192124/morse-theory/?q=2020/04871-0}{Subvention FAPESP 2020/04871-0}}

\section{\enspace Publications}
\cventry{2025}{A transcendental non-Archimedean Calabi--Yau Theorem with applications to the cscK problem}{en collaboration avec David Witt Nyström}{prépublication}{\href{https://arxiv.org/abs/2509.09442}{arXiv:2509.09442}}{Description : Nous résolvons l'équation de Monge-Ampère non archimédienne pour les variétés kähleriennes compactes, généralisant un résultat de Boucksom--Jonsson au cadre transcendant. Cela nous permet d'obtenir un critère valuatif pour la K-stabilité uniforme des modèles. Comme application principale, nous prouvons que cette condition de stabilité implique l'existence d'une métrique kählerienne à courbure scalaire constante.}
\cventry{2024}{A non-Archimedean theory of complex spaces and the csc-K problem}{publié}{\href{https://doi.org/10.1016/j.aim.2025.110543}{Advances in Mathematics}}{}{Description : Dans cet article, je construis un espace de valuations attaché à une variété complexe compacte, qui généralise l'analytification de Berkovich d'une variété projective sur $\mathds C$. Comme application, j'obtiens une condition algébrique suffisante pour l'existence de métriques kähleriennes à courbure scalaire constante sur une classe kählerienne donnée d'une variété kählerienne.}


\section{\enspace Activités professionnelles}
\subsection{Services académiques}
\cventry{2025}{Participation au cycle \href{https://smf.emath.fr/node/28305}{« Un texte, une aventure mathématique »}}{Lycée international de Saint-Germain-en-Laye}{\href{https://drive.proton.me/urls/HB78HPR8F4\#PeYTRhPIGCe4}{Diapositives et supports}}{Saint-Germain-en-Laye, France}{Description : Donné une « pré-conférence » dans un lycée préparant à la conférence \href{https://smf.emath.fr/evenements-smf/conference-bnf-b-schapira-2025}{« Dennis Sullivan, mathématicien des analogies »} donnée par Barbara Schapira. « Un texte, une aventure mathématique » est une organisation conjointe de la \href{https://smf.emath.fr/}{Société Mathématique de France} et de la \href{http://www.bnf.fr/fr/acc/x.accueil.html}{Bibliothèque nationale de France} visant à exposer le grand public à l'histoire et aux problématiques de la recherche mathématique actuelle.}
\cventry{2021}{Organisateur du \href{https://www.ime.usp.br/~acinco/en.html}{Groupe A5}}{Groupe académique géré par des étudiants}{\emph{Instituto de Matemática e Estatística}}{Universidade de São Paulo, Brésil}{Plus de \href{https://www.ime.usp.br/~acinco/past-seminars.html}{25 séminaires hebdomadaires} organisés}
\cventry{2019}{Représentant étudiant suppléant au Conseil de Département de Mathématiques}{\emph{Instituto de Matemática e Estatística}}{Universidade de São Paulo, Brésil}{}{}
\cventry{2018}{Représentant étudiant suppléant au Conseil de Département de Mathématiques}{\emph{Instituto de Matemática e Estatística}}{Universidade de São Paulo, Brésil}{}{}

\subsection{Enseignement}
\cventry{2025}{Introduction aux Surfaces de Riemann}{\emph{Campus Pierre et Marie Curie}}{Sorbonne Université, France}{}{}
\cventry{2024}{Topologie et Calcul Différentiel I}{\emph{Campus Pierre et Marie Curie}}{Sorbonne Université, France}{}{}
\cventry{2024}{Analyse Fonctionnelle}{\emph{Campus Pierre et Marie Curie}}{Sorbonne Université, France}{}{}
\cventry{2023}{Topologie et Calcul Différentiel I}{\emph{Campus Pierre et Marie Curie}}{Sorbonne Université, France}{}{}
\cventry{2020}{Introduction à la Topologie Algébrique}{\emph{Instituto de Matemática e Estatística}}{Universidade de São Paulo, Brésil}{}{}
\cventry{2019}{Introduction à l'Analyse}{\emph{Instituto de Matemática e Estatística}}{Universidade de São Paulo, Brésil}{}{}
\cventry{2019}{Fonctions Dérivables et Séries}{\emph{Instituto de Matemática e Estatística}}{Universidade de São Paulo, Brésil}{}{}
\cventry{2018}{Calcul Différentiel II}{Escola Politecnica}{Universidade de S\~ao Paulo, Brésil}{}{}

\subsection{Événements organisés}
\cventry{2022}{\href{https://www.ime.usp.br/~acinco/dias.html}{MaThematical Days : Paul Erd\H{o}s}}{Universidade de São Paulo}{São Paulo}{Brésil}{}
\cventry{2021}{\href{https://www.ime.usp.br/~acinco/noether.html}{MaThematical Days : Emmy Noether}}{Universidade de São Paulo}{São Paulo}{Brésil}{}
\cventry{2021}{\href{https://www.ime.usp.br/~acinco/cartan.html}{MaThematical Days : Élie Cartan}}{Universidade de São Paulo}{São Paulo}{Brésil}{}


\subsection{Visites scientifiques}
\cventry{2025}{\href{https://erdoscenter.renyi.hu/articles/complex-manifolds}{Simons Semester in Analysis and Geometry on Complex Manifolds}}{Erdős Center}{Budapest, Hongrie}{}{Visite de trois mois}
\cventry{2024}{\href{https://www.slmath.org/programs/361}{Special Geometric Structures and Analysis}}{SLMath}{Berkeley}{}{Visite de deux mois}





%\subsection{Autres rédactions}
%\cventry{2025}{Autour de la conjecture de Yau–Tian–Donaldson pour une classe transcendante}{Thèse de doctorat}{Sorbonne Université}{}{}
%\cventry{2022}{$L^2$ techniques in Complex Geometry}{Mémoire de master}{Sorbonne Université}{}{}
%\cventry{2021}{Morse Theory}{Mémoire de licence}{Universidade de São Paulo}{}{}

\section{\enspace Exposés}
\subsection{$\widehat K$-stabilité sur les variétés kähleriennes}
\cventry{2026}{\href{https://sites.google.com/view/k-stability-2601}{Workshop on K-stability}}{\emph{East China Normal University}} {Shanghai, Chine}{}{}
\subsection{L'approche non-archimédienne de la conjecture de Yau-Tian-Donaldson}
\cventry{2026}{\href{https://en.igp.ustc.edu.cn/2025/1230/c29595a718263/page.htm}{Geometry and topology seminar series}}{\emph{University of Science and Technology of China}} {Hefei, Chine}{}{}
\cventry{2025}{\href{https://cmls.ip-paris.fr/recherche/geometrie-et-dynamique/seminaire-de-geometrie}{Séminaire de géométrie}}{\emph{Centre de mathématiques Laurent-Schwartz}} {Palaiseau, France}{}{}
\cventry{2025}{\href{https://erdoscenter.renyi.hu/events/workshop-singular-canonical-kahler-metrics-compact-and-non-compact-manifolds}{Workshop on Singular canonical Kähler metrics on compact and non-compact manifolds}}{\emph{Rényi Institute}} {Budapest, Hongrie}{}{}
\cventry{2025}{\href{https://www.mis.mpg.de/events/event/a-non-archimedean-approach-to-the-yau-tian-donaldson-conjecture}{Geometry seminar}}{\emph{Max Plank Institute for Mathematics in the Sciences}} {Leipzig, Allemagne}{}{}
\cventry{2025}{Complex geometry seminar}{\emph{Mathematisches Institut, University of Münster}} {Munster, Allemagne}{}{}
\cventry{2025}{\href{https://delcroix.perso.math.cnrs.fr/MARGE3/}{MARGE in Brest: Fibrations and Deformations}}{\emph{Université de Bretagne Occidentale}} {Brest, France}{}{}
\cventry{2025}{\href{https://researchseminars.org/talk/AmSurAmSulGeometry/92/}{Geometry Webinar AmSur /AmSul}}{} {Webinar}{}{}
\cventry{2025}{Séminaire de Géométrie complexe}{\emph{Université Toulouse III - Paul Sabatier}} {Toulouse, France}{}{}
\cventry{2025}{\href{https://indico.math.cnrs.fr/event/13949/}{Séminaire Géométries ICJ}}{\emph{Université Claude Bernard Lyon 1}} {Lyon, France}{}{}
\cventry{2025}{\href{https://math.au.dk/en/currently/activities/event/activity/6735?cHash=aa23bc1a67618f15a8ae2123d7c7b4a3}{Mathematics Seminar}}{\emph{Aarhus University}} {Aarhus, Danemark}{}{}
\cventry{2024}{\href{https://www.imj-prg.fr/tga/sem-ga/archives.html}{Séminaire de géométrie algébrique de Jussieu}}{\emph{ENS}} {Paris, France}{}{}
\cventry{2024}{\href{https://math.univ-angers.fr/seminaires/seminaires-systemes-dynamiques-et-geometrie/}{Séminaire systèmes dynamiques et géométrie}}{\emph{ Université d'Angers}} {Angers, France}{}{}
\cventry{2024}{Complex Analysis Seminar}{\emph{University of Bayreuth}} {Bayreuth, Allemagne}{}{}
\cventry{2024}{\href{"https://www.math.pku.edu.cn/teachers/jianchunchu/seminar.html}{Geometric Analysis Seminar}}{\emph{BICMR}} {Beijing, Chine}{}{}
\cventry{2024}{\href{https://www.chalmers.se/en/current/calendar/mv-kass-seminar-pietro-mesquita-piccione/}{KASS Seminar}}{\emph{Chalmers University}} {Göteborg, Suède}{}{}
\cventry{2024}{Algebraic Geometry Seminar}{\emph{University of Glasgow}} {Écosse}{}{}
\cventry{2024}{Any Complex Geometry Seminar}{\emph{University of California Berkeley}} {États-Unis}{}{}
\cventry{2024}{\href{https://www.slmath.org/seminars/28396}{Graduate Student Seminar Series}}{\emph{SLMath}} {Berkeley, États-Unis}{}{}
\cventry{2024}{Séminaire des Thésards}{\emph{IMJ-PRG}}{Université Paris Cité, France}{}{}
\cventry{2024}{\href{https://www.imj-prg.fr/tga/demi-journee-des-doctorant-e-s-de-lequipe-tga-2024/}{Demi-Journée des doctorant.e.s de l'équipe TGA 2024}}{\emph{IMJ-PRG}}{Sorbonne Université, France}{}{}

\subsection{Un contre example a la conjecture de l'orbite périodique}
\cventry{2022}{Groupe de Travail : Dynamiques Sauvages}{\emph{IMJ-PRG}}{Sorbonne Université, France}{}{}
\cventry{2022}{A5 seminar}{\emph{Instituto de Matemática e Estatística}}{Universidade de São Paulo, Brésil}{}{}

\subsection{Grand public}
\cventry{2021}{Un conte géométrique}{S4}{\emph{Instituto de Matemática e Estatística}}{Universidade de São Paulo, Brésil}{}{}
\cventry{2020}{Théorie de Morse en 15 minutes}{Mémoire de License}{\emph{Instituto de Matemática e Estatística}}{Universidade de São Paulo, Brésil}{}{}
\cventry{2020}{La Terre est-elle plate ? Un exposé sur la courbure}{S4}{\emph{Instituto de Matemática e Estatística}}{Universidade de São Paulo, Brésil}{}{}
\cventry{2020}{Comment résoudre toutes les équations, avec certaines contraintes...}{S4}{\emph{Instituto de Matemática e Estatística}}{Universidade de São Paulo, Brésil}{}{}



\section{\enspace Autres informations}
\subsection{\enspace Activités complémentaires}
\cventry{2025}{Groupe de Travail sur le flot de Kähler Ricci}{}{}{En ligne}{}
\cventry{2021 -- 2022}{Groupe de Travail : Dynamiques Sauvages}{Groupe De Travail Dynamiques Sauvages}{IMJ-PRG}{Sorbonne Université}{}

\subsection{Langues}
\cvitemwithcomment{Portugais}{Langue maternelle}{}
\cvitemwithcomment{Italien}{Langue maternelle}{}
\cvitemwithcomment{Anglais}{Courant}{}
\cvitemwithcomment{Espagnol}{Courant}{}
\cvitemwithcomment{Français}{Courant}{}
\cvitemwithcomment{Catalan}{Intermédiaire}{}

\subsection{Liens}
\cvitem{ArXiv}{\href{https://arxiv.org/a/mesquitapiccione_p_1.html}{https://arxiv.org/a/mesquitapiccione\_p\_1}}{}{}{}{}
\cvitem{Lattes}{\href{http://lattes.cnpq.br/7822745238118143}{http://lattes.cnpq.br/7822745238118143}}{}{}{}{}
\cvitem{Orcid}{\href{https://orcid.org/0000-0002-1626-0726}{https://orcid.org/0000-0002-1626-0726}}{}{}{}{}

\subsection{Références}
\cvitem{Référence I}{\href{http://sebastien.boucksom.perso.math.cnrs.fr/}{Sébastien Boucksom}}{}{}{}{}
\cvitem{Référence II}{\href{https://www.chalmers.se/en/persons/wittnyst/}{David Witt Nyström}}{}{}{}{}
\cvitem{Référence III}{\href{https://math.umd.edu/~tdarvas/}{Tamás Darvas}}{}{}{}{}
\cvitem{Référence IV}{\href{https://finski.info/}{Siarhei Finski}}{}{}{}{}




\end{document}
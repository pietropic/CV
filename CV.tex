% CV Pietro Mesquita Piccione
% Lats updated: 02.04.2020


\documentclass[11pt,a4paper,sans]{moderncv}
\fancyfoot[l]{\parbox[b]{10cm}{\color{color2}\addressfont\strut Pietro{}~Mesquita Piccione{}~~\emailsymbol\emaillink{pietropic@pm.me}}}
\usepackage[utf8]{inputenc}
% Font sizes: 10, 11, or 12;
% paper sizes: a4paper, letterpaper, a5paper, legalpaper, executivepaper or landscape;
% font families: sans or roman

\usepackage[polutonikogreek,english]{babel}
% This is used to produce Greek alphabet in \quote



\usepackage{dsfont}

\moderncvstyle{classic}
% CV theme - options include: 'casual' (default), 'classic', 'oldstyle' and 'banking'

\moderncvcolor{orange}
% CV color - options include: 'blue' (default), 'orange', 'green', 'red', 'purple', 'grey' and 'black'

% \usepackage{lipsum}
% Used for inserting dummy 'Lorem ipsum' text into the template
% Usage: \lipsum[1-5]

\usepackage[scale=0.9]{geometry}
% Reduce document margins
%\setlength{\hintscolumnwidth}{3cm} % Uncomment to change the width of the dates column
%\setlength{\makecvtitlenamewidth}{10cm} % For the 'classic' style, uncomment to adjust the width of the space allocated to your name


%----------------------------------------------------------------------------------------
%	NAME AND CONTACT INFORMATION SECTION
%----------------------------------------------------------------------------------------

\firstname{Pietro}
\familyname{Mesquita Piccione}

% All information in this block is optional, comment out any lines you don't need
\title{Résumé}
\email{pietropic@pm.me}
\homepage{pietropic.github.io/}

% The first argument is the url for the clickable link,
% the second argument is the url displayed in the template
% this allows special characters to be displayed such as the tilde in this example

% \extrainfo{additional information}
% This will appear in the bottom of the page

%\photo[70pt][0.4pt]{pietro_2024-2.jpeg}
% The first bracket is the picture height,
% the second is the thickness of the frame around the picture (0pt for no frame)

% \quote{\bcode{*Gnw=qi seauto/n}-- Temple of Apollo at Delphi}
% \href does not seem to work inside \quote command
% http://en.wikipedia.org/wiki/Delphi#Temple_of_Apollo

%----------------------------------------------------------------------------------------

\begin{document}

% \setcounter{page}{3}

\makecvtitle % Print the CV title

\section{\enspace Education}
\cventry{2022 -- 2025}{\emph{PhD in Mathematics}}{Institut des mathématiques de Jussieu}{Sorbonne Université}{Paris, France}{Advisors: Sébastien Boucksom and Tat Dat Tô}{}
\cventry{2021 -- 2022}{\emph{Master's Degree in Mathematics}}{Institut des mathématiques de Jussieu}{Sorbonne Université}{Paris, France}{}{}
\cventry{2017 -- 2021}{\emph{Bachelor's degree in  Mathematics}}{Universidade de São Paulo}{USP}{São Paulo, Brazil}{Prize for outstanding performance.}{}




% Arguments not required can be left empty

\section{\enspace Research projects and grants}
\cventry{2022 -- 2025}{MathInParis2020}{ Scholarship for a PhD in Paris}{Cofunded by Marie Sklodowska‑Curie Actions}{FSMP}{}
\cventry {2021 -- 2022}{PGSM Scholarship}{Scholarship for a Master's degree in Paris}{Fondation Sciences mathématiques de Paris}{}{}
\cventry{2020 -- 2021}{Morse Theory}{Undergraduate Research Project}{FAPESP}{}{Grant FAPESP 2020/04871-0}
%\cventry{2020 -- 2021}{Student Study Group in PDEs}{Variational Methods in PDEs}{Undergraduate Student Activity}{\emph{Instituto de Matemática e Estatística}}{}

\section{\enspace Articles}
\cventry{2024}{A non-Archimedean theory of complex spaces and the csc-K problem}{preprint}{\href{https://arxiv.org/abs/2409.06221}{arXiv:2409.06221}}{}{}

\section{\enspace Talks}
\subsection{The non-Archimedean approach of the Yau--Tian--Donaldson conjecture}
\cventry{2024}{\href{https://www.chalmers.se/en/current/calendar/mv-kass-seminar-pietro-mesquita-piccione/}{KASS Seminar}}{\emph{Chalmers University}} {Gothenburg, Sweden}{}{}
\cventry{2024}{Algebraic Geometry Seminar}{\emph{University of Glasgow}} {Scotland}{}{}
\cventry{2024}{Any Complex Geometry Seminar}{\emph{University of California Berkeley}} {US}{}{}
\cventry{2024}{\href{https://www.slmath.org/seminars/28396}{Graduate Student Seminar Series}}{\emph{SLMath}} {Berkeley, US}{}{}
\cventry{2024}{Séminaire des Thésards}{\emph{IMJ-PRG}}{Université Paris Cité, France}{}{}
\cventry{2024}{\href{https://www.imj-prg.fr/tga/demi-journee-des-doctorant-e-s-de-lequipe-tga-2024/}{Demi-Journée des doctorant.e.s de l'équipe TGA 2024}}{\emph{IMJ-PRG}}{Sorbonne Université, France}{}{}
\subsection{A counterexample for the periodic orbit conjeture}
\cventry{2022}{Groupe de Travail : Dynamiques Sauvages}{\emph{IMJ-PRG}}{Sorbonne Universite, France}{}{}
\cventry{2022}{A5 seminar}{\emph{Instituto de Matemática e Estatística}}{Universidade de São Paulo, Brasil}{}{}

\subsection{To the general public}
\cventry{2021}{A Tale of Geometry}{S4}{\emph{Instituto de Matemática e Estatística}}{Universidade de São Paulo, Brazil}{}{}
\cventry{2020}{Morse Theory in 15 Minutes}{Senior Thesis Presentation}{\emph{Instituto de Matemática e Estatística}}{Universidade de São Paulo, Brazil}{}{}
\cventry{2020}{Is the Earth flat? A talk about curvature}{S4}{\emph{Instituto de Matemática e Estatística}}{Universidade de São Paulo, Brazil}{}{}
\cventry{2020}{How to solve any equation, with some conditions}{S4}{\emph{Instituto de Matemática e Estatística}}{Universidade de São Paulo, Brazil}{}{}







%----------------------------------------------------------------------------------------
%	Work SECTION
%----------------------------------------------------------------------------------------

\newpage

\section{\enspace Professional}
\subsection{Visits}
\cventry{2024}{\href{https://www.slmath.org/programs/361}{Special Geometric Structures and Analysis}}{SLMath}{Berkeley}{}{Two months visit to the program.}

\subsection{Events attended}
\cventry{2024}{Summer school on Complex Geometry and canonical metrics}{CRM}{Montréal}{Canada}{}
\cventry{2023}{Global invariants of arithmetic varieties}{CIRM}{Luminy}{France}{}
\cventry{2023}{Summer School and Worskhop "Advances in special Kähler
metrics"}{}{Le Croisic}{France}{}
\cventry{2023}{Complex analysis and geometry: celebrating the 70+1th birthday of László Lempert}{Rényi Institute}{Budapest}{Hungary}{}
\cventry{2022}{Géométrie : échanges et perspectives}{Insitut Henri Poincaré}{Paris}{France}{}
\cventry{2021}{Géométrie : échanges et perspectives}{Insitut Henri Poincaré}{Paris}{France}{}
\cventry{2019}{32nd Brazilian Mathematics Colloquium}{IMPA}{Rio de Janeiro}{Brazil}{}
\cventry{2018}{ICM 2018}{International Congress of Mathematicians}{Rio de Janeiro}{Brazil}{}{}
\subsection{Events Organized}
\cventry{2022}{\href{https://www.ime.usp.br/~acinco/dias.html}{MaThematical Days : Paul Erdös}}{Universidade de São Paulo}{São Paulo}{Brazil}{}
\cventry{2021}{\href{https://www.ime.usp.br/~acinco/noether.html}{MaThematical Days : Emmy Noether}}{Universidade de São Paulo}{São Paulo}{Brazil}{}
\cventry{2021}{\href{https://www.ime.usp.br/~acinco/cartan.html}{MaThematical Days : Élie Cartan}}{Universidade de São Paulo}{São Paulo}{Brazil}{}


\subsection{Teaching Assistantships}
%\subsubsection{Teaching Assistantship}
\cventry{2025/1}{Introduction to Riemann Surfaces}{\emph{Campus Pierre et Marie Curie}}{Sorbonne Université, France}{}{}{}
\cventry{2024/2}{Topology and Differentiable Calculus I}{\emph{Campus Pierre et Marie Curie}}{Sorbonne Université, France}{}{}{}
%\cventry{2020 -- current}{Private Tutoring}{Mathematical Analysis}{}{S\~ao Paulo, Brazil}{}
\cventry{2024/1}{Functional Analysis}{\emph{Campus Pierre et Marie Curie}}{Sorbonne Université, France}{}{}{}
\cventry{2023/2}{Topology and Differentiable Calculus I}{\emph{Campus Pierre et Marie Curie}}{Sorbonne Université, France}{}{}{}
\cventry{2020/2}{Introduction to Algebraic Topology}{\emph{Instituto de Matemática e Estatística}}{Universidade de São Paulo, Brazil}{}{}{}
\cventry{2019/2}{Introduction to Analysis}{\emph{Instituto de Matemática e Estatística}}{Universidade de São Paulo, Brazil}{}{}{}
\cventry{2019/1}{Differentiable functions and series}{\emph{Instituto de Matemática e Estatística}}{Universidade de São Paulo, Brazil}{}{}{}
\cventry{2018/2}{Calculus II}{Escola Politecnica}{Universidade de S\~ao Paulo, Brazil}{}{}


\subsection{Academic Services}
\cventry{2021}{Organizer of the \href{https://www.ime.usp.br/~acinco/en.html}{A5 Group}}{A student runned academic group}{\emph{Instituto de Matemática e Estatística}}{Universidade de São Paulo, Brazil}{Over \href{https://www.ime.usp.br/~acinco/past-seminars.html}{25 weekly seminars} organized}
\cventry{2019}{Deputy Student Representative in the Mathematics Department Council}{\emph{Instituto de Matemática e Estatística}}{Universidade de São Paulo, Brazil}{}{}
\cventry{2018}{Deputy Student Representative in the Mathematics Department Council}{\emph{Instituto de Matemática e Estatística}}{Universidade de São Paulo, Brazil}{}{}

\subsection{Other Writings}
\cventry{2022}{$L^2$ techniques in Complex Geometry}{Master Thesis}{Sorbonne Université}{}{}
\cventry{2021}{Morse Theory}{Undergraduate Thesis}{Universidade de São Paulo}{}{}





\section{\enspace Other Info}
\subsection{\enspace Related Activities}
\cventry{2021 -- 2022}{Reading Seminar on Wild Dynamics}{Groupe De Travail Dynamiques Sauvages}{IMJ-PRG}{Sorbonne Université}{}

\cventry{2021}{Summer IMPA}{Differential topology graduate course}{IMPA}{Rio de Janeiro, Brazil}{Online}

\cventry{2020}{Summer IMPA}{Functional Analysis graduate course}{IMPA}{Rio de Janeiro, Brazil}{with financial support}

\cventry{2018}{Summer IMPA}{Real Analysis}{IMPA}{Rio de Janeiro, Brazil}{with financial support}

\subsection{Languages}

\cvitemwithcomment{Portuguese}{Native Speaker}{}
\cvitemwithcomment{Italian}{Native Speaker}{}
\cvitemwithcomment{English}{Fluent}{}
\cvitemwithcomment{Spanish}{Fluent}{}
\cvitemwithcomment{French}{Fluent}{}
\cvitemwithcomment{Catalan}{Intermediate}{}


\subsection{Links}
\cvitem{Lattes}{\href{http://lattes.cnpq.br/7822745238118143}{http://lattes.cnpq.br/7822745238118143}}{}{}{}{}
\cvitem{Orcid}{\href{https://orcid.org/0000-0002-1626-0726}{https://orcid.org/0000-0002-1626-0726}}{}{}{}{}

\subsection{References}
\cvitem{Reference I}{\href{http://sebastien.boucksom.perso.math.cnrs.fr/}{Sébastien Boucksom}}{}{}{}{}
\cvitem{Reference II}{\href{https://sites.google.com/site/totatdatmath/home}{Tat Dat Tô}}{}{}{}{}
\cvitem{Reference III}{\href{https://math.umd.edu/~tdarvas/}{Tamás Darvas}}{}{}{}{}
\cvitem{Reference IV}{\href{https://www.maths.gla.ac.uk/~rdervan/}{Ruadhaí Dervan}}{}{}{}{}
\end{document}
